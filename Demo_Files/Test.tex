\documentclass[twocolumn,twoside]{IEEEtran}                 %% twocolumn (regular paper)
%\documentclass[twocolumn,twoside,dvipdfm,technote]{IEEEtran}        %% twocolumn (correspondence)
%\documentclass[onecolumn,twoside,draft,dvipdfm,technote]{IEEEtran}  %% onecolumn,draft (correspondence)
%\documentclass[onecolumn,twoside,draft,dvipdfm]{ieeetran}            %% onecolumn,draft (regular paper)
%\documentclass[12pt,journal,draftclsnofoot,onecolumn,letterpaper]{IEEEtran} %TWC
%\textwidth= 6.5in      % 6.5in = letterpaper
%\textheight = 8.5in   % 8.5in = 29 line

\usepackage{amssymb}
\usepackage{amsmath}
\usepackage{amsmath,bm}
\usepackage{amsthm}
\usepackage{graphicx}
\usepackage{subfigure}
\usepackage{cite}
\usepackage{enumerate}
%\usepackage[ruled]{algorithm2e}
\usepackage{color, soul}
\usepackage{algorithm}
\usepackage{algorithmic}

\begin{document}


% ==================================================
\title{UAV Sensing Implemented Indoor 3D Wi-Fi Spectrum Rebuilding}

\author{Yuzhe~Yang, %~\IEEEmembership{Member,~IEEE,}
        Chao~Yao, %~\IEEEmembership{Fellow,~OSA,}
        and~Lingyang~Song,~\IEEEmembership{Life~Fellow,~IEEE}% <-this % stops a space
\thanks{M. Shell is with the Department
of Electrical and Computer Engineering, Georgia Institute of Technology, Atlanta,
GA, 30332 USA e-mail: (see http://www.michaelshell.org/contact.html).}% <-this % stops a space
\thanks{J. Doe and J. Doe are with Anonymous University.}% <-this % stops a space
\thanks{Manuscript received April 19, 2005; revised January 11, 2007.}}

\markboth{Journal of \LaTeX\ Class Files,~Vol.~6, No.~1, January~2007}%
{Shell \MakeLowercase{\textit{et al.}}: Bare Demo of IEEEtran.cls for Journals}

\maketitle


% ==================================================
\begin{abstract}
%\boldmath
Nowadays UAV is the most convenient things to implement spectrum seing in indoor or out door circumstance.\
This statement aims to test the "enter" is allright.\
So that's why we talk about this.
\end{abstract}

\begin{IEEEkeywords}
IEEEtran, journal, \LaTeX, paper, template.
\end{IEEEkeywords}


% ==================================================
\section{Introduction}

\IEEEPARstart{T}{his} demo file is intended to serve as a ``starter file''
for IEEE journal papers produced under \LaTeX\ using
IEEEtran.cls version 1.7 and later.

I wish you the best of success.

These paper is created by $Yang$ $Yuzhe$ for demo.

Hope it can help you in some ways.

Also, bless for my own paper. Fight on.

\hfill Yuzhe Yang

\hfill December 27, 2016


\subsection{Subsection Heading Here}
Subsection text here.


\subsubsection{Subsubsection Heading Here}
Subsubsection text here.



% ==================================================
\section{Conclusion}
The conclusion goes here.

\begin{algorithm}[h]
\caption{UAV Sensing for the $t_{i}$-th survey}
\begin{algorithmic}[1]
\STATE (1) //During the first measurement period ($method$ $one$)
\FOR{$j=1$ to $m$}
\STATE Measure spectrum merits and record;
\STATE Move and randomize current locations within $Cube_{i}$;
\ENDFOR

\STATE
\STATE (2) //During a navigation period
\STATE $Cube_{next}\gets $ determine next Cube under present $Cube_{i}$;
\IF {$Cube_{next}$ == NULL }
\STATE move to a start-point;
\STATE enter the update period(3);
\ELSE
\STATE move to $Cube_{next}$;
\STATE enter the measurement period(1);
\ENDIF

\STATE
\STATE (3) //During the update period
\STATE update a spectrum-condition map under $Cube_{i}$;
\STATE $c_{i}\gets $ count cubes whose condition deviates by $\sigma$;
\IF {$c_{i} > 0$ and $Curmode$ == $method$ $two$}
\STATE add $Cube_{i}$ to $SuspectCubes(E_{n})$;
\FOR{$j=1$ to $neighbor$ $cubes$}
\STATE enter the measurement period(1);
\ENDFOR
\STATE update the recent spectrum map;
\ENDIF

\STATE
\STATE (4) //During the selective period ($method$ $three$)
\FORALL {$c_{i}$ such that $c_{i} \in SuspectCubes(E_{n})$}
\STATE Generate Min Coverage $D$ of $SuspectCubes(E_{n})$;
\ENDFOR
\FORALL {$c_{i} \in D$}
\IF {$c_{i} \gets $ count deviates by $\sigma$}
\STATE enter the update period(3);
\ENDIF
\ENDFOR

\end{algorithmic}
\end{algorithm}


% ==================================================
\appendices
\section{Proof of the First Zonklar Equation}
Appendix one text goes here.


\section{}
Appendix two text goes here.


\section*{Acknowledgment}

The authors would like to thank...

\ifCLASSOPTIONcaptionsoff
  \newpage
\fi



% ==================================================
\begin{thebibliography}{1}

\bibitem{IEEEhowto:kopka}
Chen H. C., H. T. Kung, D. Vlah, D. Hague, M. Muccio, and B. Poland. “Collaborative Compressive Spectrum Sensing in a UAV Environment.” In 2011 - MILCOM 2011 Military Communications Conference, 142–48, 2011.

\bibitem{IEEEhowto:kopka}
Hingu V., and S. Shah. “Block-Wise Eigenvalue Based Spectrum Sensing Algorithm in Cognitive Radio Network.” In 2015 9th Asia Modelling Symposium (AMS), 85–88, 2015.

\bibitem{IEEEhowto:kopka}
Xue Haozhou, and Feifei Gao. “A Machine Learning Based Spectrum-Sensing Algorithm Using Sample Covariance Matrix.” In 2015 10th International Conference on Communications and Networking in China (ChinaCom), 476–80, 2015.

\bibitem{IEEEhowto:kopka}
Ye Fang, and Xun Zhang. “Non-Uniform Quantized Exponential Entropy-Based Spectrum Sensing Algorithm in Cognitive Radio.” In 2016 Progress in Electromagnetic Research Symposium (PIERS), 2511–16, 2016.

\end{thebibliography}


\begin{IEEEbiography}{Michael Shell}
Biography text here.
\end{IEEEbiography}

% if you will not have a photo at all:
\begin{IEEEbiographynophoto}{John Doe}
Biography text here.
\end{IEEEbiographynophoto}


\begin{IEEEbiographynophoto}{Jane Doe}
Biography text here.
\end{IEEEbiographynophoto}

\end{document}
