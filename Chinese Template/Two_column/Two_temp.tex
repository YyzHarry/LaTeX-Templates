\documentclass[a4paper,11pt,onecolumn,twoside]{article}

\usepackage{xeCJK}       % 使用XeLaTeX编译
\usepackage{CJK}         % CJK中文支持
\usepackage{fancyhdr}
\usepackage{amsmath,amsfonts,amssymb,graphicx}    % EPS图片支持
\usepackage{graphics}
\usepackage{subfigure}   % 使用子图形
\usepackage{indentfirst} % 中文段落首行缩进
\usepackage{bm}          % 公式中的粗体字符(用命令\boldsymbol)
\usepackage{multicol}    % 正文双栏
\usepackage{indentfirst} % 中文首段缩进
%\usepackage{picins}     % 图片嵌入段落宏包 比如照片
\usepackage{abstract}    % 2栏文档,一栏摘要及关键字宏包

%%%%%%%%%%%%%%%%%%%%%%%%%%%%%%%%%%%%%%%%%%%%%%%%%%%%%%%%%%%%%%%%
%  lengths
%  下面的命令重定义页面边距,使其符合中文刊物习惯。
%%%%%%%%%%%%%%%%%%%%%%%%%%%%%%%%%%%%%%%%%%%%%%%%%%%%%%%%%%%%%%%%

\addtolength{\topmargin}{-54pt}
\setlength{\oddsidemargin}{-0.9cm}  % 3.17cm - 1 inch
\setlength{\evensidemargin}{\oddsidemargin}
\setlength{\textwidth}{17.00cm}
\setlength{\textheight}{24.00cm}    % 24.62
\setCJKmainfont[BoldFont=SimHei,ItalicFont={[stkaiti.ttf]}]{SimSun}

\newfontfamily\kai{STKaiti}          % 楷体
\newfontfamily\hei{SimHei}           % 黑体

%%%%%%%%%%%%%%%%%%%%%%%%%%%%%%%%%%%%%%%%%%%%%%%%%%%%%%%%%%%%%%%%
%  定义标题格式,包括title,author,affiliation,email等。
%  在任何用到中文的地方,用\begin{CJK} ... \end{CJK}将其括起来。
%%%%%%%%%%%%%%%%%%%%%%%%%%%%%%%%%%%%%%%%%%%%%%%%%%%%%%%%%%%%%%%%

\renewcommand{\baselinestretch}{1.1} %定义行间距
\parindent 22pt %重新定义缩进长度

%%%%%%%%%%%%%%%%%%%%%%%%%%%%%%%%%%%%%%%%%%%%%%%%%%%%%%%%%%%%%%%%
% 标题,作者,通信地址定义
%%%%%%%%%%%%%%%%%%%%%%%%%%%%%%%%%%%%%%%%%%%%%%%%%%%%%%%%%%%%%%%%

\title{\huge{综述:???}
\thanks{本文为 \textbf{xxxxxx} 课程第一阶段论文}}
\author{Yyz \ \small{(23333)}\\[2pt]
\normalsize
指导老师:xxx
\\[2pt]}

\date{}  % 这一行用来去掉默认的日期显示

%%%%%%%%%%%%%%%%%%%%%%%%%%%%%%%%%%%%%%%%%%%%%%%%%%%%%%%%%%%%%%%%
% 首页页眉页脚定义
%%%%%%%%%%%%%%%%%%%%%%%%%%%%%%%%%%%%%%%%%%%%%%%%%%%%%%%%%%%%%%%%

\fancypagestyle{plain}{
\fancyhf{}
\lhead{School of EECS\\
\scriptsize{Peking University}}
\chead{\centering{综述:???\\
\scriptsize{\textbf{Summery: Emission, Propagation and Scattering of Electromagnetic Wave}}}}
\rhead{xxx\\
\scriptsize{Yuzhe Yang}}
\lfoot{}
\cfoot{}
\rfoot{}}

%%%%%%%%%%%%%%%%%%%%%%%%%%%%%%%%%%%%%%%%%%%%%%%%%%%%%%%%%%%%%%%%
% 首页后根据奇偶页不同设置页眉页脚
% R,C,L分别代表左中右,O,E代表奇偶页
%%%%%%%%%%%%%%%%%%%%%%%%%%%%%%%%%%%%%%%%%%%%%%%%%%%%%%%%%%%%%%%%

\pagestyle{fancy}
\fancyhf{}
\fancyhead[R]{Yyz xxx}
\fancyhead[C]{综述:电磁波发射、传播与散射}
\fancyhead[L]{\thepage}
\lfoot{}
\cfoot{}
\rfoot{}

%%%%%%%%%%%%%%%%%%%%%%%%%%%%%%%%%%%%%%%%%%%%%%%%%%%%%%%%%%%%%%%%
% 正文两栏环境不允许float环境,比如 figure, table。所以重新定义
% figure,使之可以浮动到你想要的位置。table也同样,把figure改为
% table就可以。
%%%%%%%%%%%%%%%%%%%%%%%%%%%%%%%%%%%%%%%%%%%%%%%%%%%%%%%%%%%%%%%%

\newenvironment{figurehere}
  {\def\@captype{figure}}
  {}
\makeatother



\begin{document}

\newcommand{\supercite}[1]{\textsuperscript{\cite{#1}}}

%  显示title,并设页码为空(按杂志社要求)
\maketitle


%  中文摘要
%  调整摘要、关键词,中图分类号的页边距
%  中英文同时调整

\setlength{\oddsidemargin}{ 1cm}  % 3.17cm - 1 inch
\setlength{\evensidemargin}{\oddsidemargin}
\setlength{\textwidth}{13.50cm}
\vspace{-.8cm}

\begin{center}
\parbox{\textwidth}{
\textbf{摘\ \ 要}\quad  这里写你的摘要。\\
\textbf{关键词}\quad  这里写上你的关键词。当然,非正式的论文可以将这两行注释掉}
\end{center}


% ======================================================Abstract

\vspace{.1cm}
\begin{center}
\parbox{\textwidth}{
{\large{\textbf{Summery: Emission, Propagation and Scattering of Electromagnetic Wave}}}\\
\vspace{-0.5cm}
\begin{center}
\textbf{Yuzhe Yang}\\[2pt]
\small{\textit{(Dept. EECS, Peking University, Beijing 100871, China)}}\\[2pt]
\end{center}
{\small{\textbf{Abstract}\quad You can write your Abstract here. \\
\textbf{Key Words}\quad You can write your key words here.}}
}
\end{center}


%  恢复正文页边距
%%%%%%%%%%%%%%%%%%%%%%%%%%%%%%%%%%%%%%%%%%%%%%%%%%%%%%%%%%%%%%%%
\setlength{\oddsidemargin}{-.5cm}  % 3.17cm - 1 inch
\setlength{\evensidemargin}{\oddsidemargin}
\setlength{\textwidth}{17.00cm}
%\CJKfamily{song}


%  分栏开始
%  单栏的可以直接把这句话注释掉(包括文末的\end)
\begin{multicols}{2}



\section{引言}

这里写你的引言。


\section{电磁波概述}

\begin{equation}
\left\{
\begin{aligned}
& \nabla \times \vec{E} = -\frac{\partial \vec{B}}{\partial t}, \\
& \nabla \cdot \vec{B} = 0, \\
& \nabla \times \vec{B} = \mu_0 \vec{\mathcal J} + \mu_0\varepsilon_0\frac{\partial \vec{E}}{\partial t}, \\
& \nabla \cdot \vec{E} = \frac{\rho}{\varepsilon_0}
\end{aligned}
\right.
\label{maxwell}
\end{equation}

某个section。


\section{电磁波的辐射}

另一个section。

\subsection{推迟势}

某个subsection。

在非静态的情况下,引入磁矢势$\vec{A}$($\vec{B} = \nabla \times \vec{A}$),带入麦克斯韦方程组可知
\begin{equation}
\vec{E} + \frac{\partial \vec{A}}{\partial t} = - \nabla \varphi
\end{equation}
此时$\vec{E}$不再是保守场,我们采用$\vec{A}$和$\varphi$来描述电磁场。引入Lorentz规范~\supercite{Elecbook},可以推导出在此规范下有:
\begin{equation}
\left\{
\begin{aligned}
& \nabla^2 \vec{A} - \frac{1}{c^2} \frac{\partial^2 \vec{A}}{\partial t^2} = -\mu_0 \mathcal{J}, \\
& \nabla^2 \varphi - \frac{1}{c^2} \frac{\partial^2 \varphi}{\partial t^2} = -\frac{\rho}{\varepsilon_0}.
\end{aligned}
\right.
\label{dbert}
\end{equation}

上述方程~\ref{dbert}被称作d'Alembert方程,是一个非齐次的波动方程。我们通过描述$\vec{A}$、$\varphi$的波动性来描述$\vec{E}$、$\vec{B}$的波动性,以此来描述电磁场。考虑源为变化的电荷/电流,解上述方程可得

\begin{equation}
\left\{
\begin{aligned}
& \varphi(\vec{x},t) = \frac{1}{4\pi \epsilon_0} \int_{V} \frac{\rho(\vec{x},t-\frac{r}{c})}{r} dV^{'}, \\
& \vec{A}(\vec{x},t) = \frac{\mu_0}{4\pi} \int_{V} \frac{\mathcal{J}(\vec{x},t-\frac{r}{c})}{r} dV^{'}.
\end{aligned}
\right.
\end{equation}


\subsection{辐射场分布}

另一个subsection。

\begin{equation}
\begin{split}
& \vec{A}(\vec{x},t) = \vec{A}(\vec{x})\exp(-i\omega t), \\
where \quad & \vec{A}(\vec{x}) = \frac{\mu_0}{4\pi} \int_{V} \frac{\mathcal{J}(\vec{x}^{'}) \exp(-ikr)}{r} dV^{'}.
\end{split}
\end{equation}


\subsection{天线}

最后一个subsection。

\begin{equation*}
P_{rad} = \int U(\theta,\phi) d\Omega = \int_{0}^{2\pi} \int_{0}^{\pi} U(\theta,\phi) \sin \theta d\theta d\phi.
\end{equation*}

\begin{equation}
F = \Bigg| \frac{\sin(\frac{N}{2}kl\cos \theta)}{\sin(\frac{1}{2}kl\cos \theta)} \Bigg|^2
\end{equation}
令$\psi = \frac{\pi}{2}-\theta$,则第一零点对应的主瓣半角宽度为
\begin{equation}
\psi = \sin^{-1} (\frac{\lambda}{Nl}) \simeq \frac{\lambda}{Nl}
\end{equation}
当$Nl\gg \lambda$时,便可以获得高度定向的辐射。

懒得把公式删掉了。


\section{电磁波的传播}

$\\itemize$用法。

\begin{itemize}
\item[$\bullet$] 距离衰减(波束扩散)
\item[$\bullet$] 大气吸收衰减
\item[$\bullet$] 陆地/海洋边界
\item[$\bullet$] 大气分层边界
\item[$\bullet$] 地形和建筑物
\end{itemize}


电磁波传播的主要途径有:
\begin{itemize}
\item[*] 天波传播:主要是利用电离层反射进行传播,波传播距离可超过1万千米。
\item[*] 地波传播:沿陆地/海洋表面传播,距离几百千米至几千千米。
\item[*] 视距传输:传播距离约为$r = \sqrt{2R} (\sqrt{H_1}+\sqrt{H_2})$,易受地面反射干扰。
\item[*] 超视距传播:利用低层大气的分层特征通过反射和折射实现超视距传播。
\end{itemize}

\section{电磁波的散射}

\subsection{电波散射概述}

懒得删了。

\subsection{数值求解: 计算电磁学}

一行行删有点麻烦。
\begin{equation}
\varphi_O = \frac{1}{4} \big(\varphi_A + \varphi_B + \varphi_C + \varphi_D + l^2\frac{\rho(O)}{\varepsilon(O)} \big),
\end{equation}

终于弄完勒。


\section{总结}

在这里写上你的总结。你会发现这篇文章中有很多地方没有被改动,因为我比较懒。

最后,由于报告中需要加入部分数学表达式,本文档使用了$LaTeX$来编写报告,以获得更优的排版效果
\footnote{You can add some footnotes here.}。


% =====================================================Reference
\small
\begin{thebibliography}{99}
\setlength{\parskip}{0pt}  %段落之间的竖直距离

\bibitem{wiki}
https://zh.wikipedia.org/wiki.

\bibitem{Elecbook}
郭硕鸿. 电动力学~(第三版), 高等教育出版社, 2008.6.

\bibitem{newton}
Newton. \emph{Philosophiæ Naturalis Principia Mathematica}. Jul. 1686.

\bibitem{chap1}
引用的格式根据要求自行定义。


\end{thebibliography}


% ===================================================Annotations
\iffalse
\normalsize
\parpic{%
\includegraphics[width=3.0cm]%
{Hou.jpg}}
\indent Yuzhe, ...
\fi


\end{multicols}


\clearpage
\end{document}
